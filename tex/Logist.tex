\documentclass{article}
\usepackage{german,a4,psfig,html,htmllist}

%\pagestyle{empty}
\pagestyle{plain}


%\textwidth 18cm
%\textheight 25cm
%\topmargin 0.3cm
%\headheight 0mm
%\headsep 1pt
%\parindent 0cm
%\oddsidemargin -1.2cm
%\evensidemargin -1.2cm


% --------------------------------------------------

\parindent=0mm
\parskip=2mm

% --------------------------------------------------

\include{../../commonTeX/all}

% --------------------------------------------------

\begin{document}

%---------------------------
%  "Ubung 1 
%
%   Einf"uhrung in Unix und NSL
%   Beispiele:
%     - logistische Gleichung (logistische Differenzen-Gleichung)
%     - didday (im Text nicht erw"ahnt. Nur, falls wir in der "Ubung
%           zu schnell voran kommen)
%---------------------------


% ====== header

%\begin{flushright} \today \end{flushright}

\begin{center}
\setcounter{section}{1}
\setcounter{equation}{0}
\setcounter{subsection}{0}
\section*{"Ubung \thesection{} zur Vorlesung\\ 
 ``Perzeption in selbstorganisierenden Nervennetzen''} 
\centerline{\large Einf"uhrung}
\large (Abgabe der diesmal allesamt `freiwilligen' Aufgaben an einem 
beliebigen Termin)
\end{center}


% ======================================================================

\subsection{Administratives}

Rechnerzeiten sind vorzugsweise: {\em Mo-Fr: 8:00-10:00 und ab 18:00 Uhr.}
\noindent 
F"ur Fragen und Beratung stehen zur Verf"ugung:

\begin{tabular}{lll}
  Name   & Raum \\
  Jochen Triesch (Inhaltliche Fragen zu den "Ubungen)   
    & ND-03 69 \\
  Jan Vorbr"uggen (Speziell morgens 8:00-10:00 Uhr)
    & ND-03 33 \\
  Christian Eckes (Speziell abends ab 18:00 Uhr)
    & ND-03 70 \\
  Michael Neef (nur {\bf dringende} Systemfragen, z.B. Rechner kaputt)  
    &  ND-03 68 \\
\end{tabular}



\subsection{Erste Schritte}

% ===
\subsubsection{Betriebssystem {\sc Unix}}

Loggen Sie sich ein. Die dazu notwendigen Loginnamen und  Pa"sw"orter 
werden in der "Ubung mitgeteilt.
Machen Sie sich mit Ihrem Dateisystem vertraut. Wichtige Kommandos:

\commandBox{
    \tt     cd {\em directoryname} \\
    cd .. \\
    ls [-l] [{\em directoryname}] \\
    pwd \\
}


% ======================================================================

\subsection{Numerisches L"osen von Differentialgleichungen (DGLs)}

\begin{itemize}

% ---
\item[(a)] Starten Sie die \htmladdnormallink{Java-Simulation Logist.}{../../Logist.html} \\
Was wird in den 6 Graphen dargestellt?\\
W"ahlen Sie andere Werte f"ur $\alpha$ ($\alpha>0$, z.B. \texttt{alpha = 3.7}).

% ---
\item[(b*)] Zum wieder warm werden mit L"osungen von DGLs (gedacht als
freiwillige `Hausaufgabe'):\\
Die zugrunde liegende {\em logistische Differentialgleichung}
\begin{eqnarray}    \label{eqnLogDGL}
  \dot{x} = {dx \over dt} = (\alpha - 1) x - \alpha x^2
\end{eqnarray}
ist noch einfach genug, um eine analytische L"osung mittels
Separation der Variablen und elementar bestimmter Integrale zuzulassen.
F"uhren Sie diese durch und diskutieren Sie das Verhalten der Fixpunkte
anhand der L"osungen.

% --- 
\item[(c)] Die DGL wird nicht analytisch sondern
numerisch berechnet. W"ahlen Sie mit z.B. \texttt{delta = 0.8}
gr"o"sere Schrittweiten f"ur die numerische Integration der DGL und
beobachten sie das Resultat.

Das hier verwendete Integrationsverfahren nach Euler f"uhrt  bei
der Schrittweite 1.0 die logistische DGL~(\ref{eqnLogDGL}) in die 
bekannte {\em logistische Diffenzengleichung}
\begin{eqnarray}    \label{eqnLogDiffGl}
  x_{n+1} = \alpha x_n (1 - x_n)\;, \qquad 0 \leq \alpha \leq 4
\end{eqnarray}
"uber. Sie ist eine der bekanntesten Gleichungen f"ur Chaos ($\alpha=3.1$
2er Periode, $\alpha=3.5$ 4er Periode, $\alpha>3.57$ chaotisch).

% --- 
\item[(d*)] Leiten sie wie beschrieben aus~(\ref{eqnLogDGL})
die Gleichung~(\ref{eqnLogDiffGl}) her.

\end{itemize}

\vspace{.5em}
{\em Dieses Beispiel soll die Schwierigkeit bei der Wahl geeigneter
Intergrationsschrittweiten beleuchten.}


% ------------------------------

\vspace{1em}

Die Aufgaben mit $\star$ sind Zus"atze f"ur die Leute, denen der Rest
zu langweilig ist. 

\vspace{1em}

Viel Spa"s!

\vspace{3em}

Die WWW-Seiten inklusive des Java-Quelltextes und der "ubersetzten
class-Dateien befinden sich als Archiv in zwei Versionen komplett zum
Downloaden auf dem Institutsserver.

Das eine Archiv ist als komprimierte tar-Datei \texttt{Logist.tar.gz} f"ur
Unix Betriebssysteme, das andere f"ur Windows 95
in der Datei \texttt{Logist.zip} verf"ugbar. Der Inhalt der Dateien ist
identisch.



\end{document}
